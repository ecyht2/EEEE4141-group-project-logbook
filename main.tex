\documentclass[12pt]{article}
\usepackage{nottingham-logbook}
\usepackage{nott-titlepage}
\usepackage{hyperref}
\usepackage{graphicx}

\title{Group Project Logbook}
\author{Tan Hong Kai}
\studentid{20386501}
\date{Year 2023 - 2024}
\department{Department of Electrical and Electronics Engineering}

\begin{document}
\maketitle

\begin{logbook-entry}{Path Generation}{27}{04/03/2024}{08/03/2024}
The path generation and points interpolation algorithm is worked on.

\subsection*{Path Generation}
A simple algorithm is used to generate the path followed by the boat.
The algorithm can be split into a few main logic.

\begin{enumerate}
    \item Pick an initial starting point of the boat (the first point inserted by the user)
    \item Find the nearest point to the initial point (using Haversine formula (provided by maplibrejs library))
    \item Add the nearest point to the path
    \item Use repeat step 2 with the new point as the new initial point
    \item The algorithm ends when all the input points are used
\end{enumerate}

The algorithm is simple and works but will sometimes generate path that intersects.
This is not ideal as this would mean the boat will collect redundant data (data in the same coordinate).
However, this is good enough for now but a better algorithm can be used.

\subsection*{Point Interpolation}
The Haversine Formula can be used to interpolate the points in between the path.
This can be done by using the bearing from the staring to the ending point of the path and the distance from the starting point.
The equations are shown below:

New Latitude:
\begin{equation}
    \phi{}_{2} = \arcsin{(\sin{\phi{}_{1}} \cos{\delta} + \cos{\phi{}_{1}} \sin{\delta} \cos{\theta})}
\end{equation}

New Longitude:
\begin{equation}
    \lambda{}_{2} = \lambda{}_{1} + \arctan{(\frac{\sin{\theta} \sin{\delta} \cos{\phi{}_{1}}}{\cos{\delta} - \sin{\phi_{1} \sin{\phi_{2}}}})}
\end{equation}

Variables:

$\delta = d / R$

$d = $ distance from the starting point

$R = $ the radius of Earth

$\phi_1 = $ The latitude of the starting point

$\lambda_1 = $ The longitude of the starting point

$\theta = $ The bearing from north

\subsection*{Resources}
\begin{itemize}
    \item \href{https://stackoverflow.com/questions/14263284/create-non-intersecting-polygon-passing-through-all-given-points}{Better Path Generation}
    \item \href{https://www.movable-type.co.uk/scripts/latlong.html}{Haversine Formula}.
\end{itemize}
\end{logbook-entry}

\end{document}
