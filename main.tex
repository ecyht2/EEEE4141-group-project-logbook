\documentclass[12pt]{article}
\usepackage{nottingham-logbook}
\usepackage{nott-titlepage}
\usepackage{hyperref}
\usepackage{graphicx}

\title{Group Project Logbook}
\author{Tan Hong Kai}
\studentid{20386501}
\date{Year 2023 - 2024}
\department{Department of Electrical and Electronics Engineering}

\begin{document}
\maketitle

\begin{logbook-entry}{Path Generation}{27}{04/03/2024}{08/03/2024}
The path generation and points interpolation algorithm is worked on.

\subsection*{Path Generation}
A simple algorithm is used to generate the path followed by the boat.
The algorithm can be split into a few main logic.

\begin{enumerate}
    \item Pick an initial starting point of the boat (the first point inserted by the user)
    \item Find the nearest point to the initial point (using Haversine formula (provided by maplibrejs library))
    \item Add the nearest point to the path
    \item Use repeat step 2 with the new point as the new initial point
    \item The algorithm ends when all the input points are used
\end{enumerate}

The algorithm is simple and works but will sometimes generate path that intersects.
This is not ideal as this would mean the boat will collect redundant data (data in the same coordinate).
However, this is good enough for now but a better algorithm can be used.

\subsection*{Point Interpolation}
The Haversine Formula can be used to interpolate the points in between the path.
This can be done by using the bearing from the staring to the ending point of the path and the distance from the starting point.
The equations are shown below:

New Latitude:
\begin{equation}
    \phi{}_{2} = \arcsin{(\sin{\phi{}_{1}} \cos{\delta} + \cos{\phi{}_{1}} \sin{\delta} \cos{\theta})}
\end{equation}

New Longitude:
\begin{equation}
    \lambda{}_{2} = \lambda{}_{1} + \arctan{(\frac{\sin{\theta} \sin{\delta} \cos{\phi{}_{1}}}{\cos{\delta} - \sin{\phi_{1} \sin{\phi_{2}}}})}
\end{equation}

Variables:

$\delta = d / R$

$d = $ distance from the starting point

$R = $ the radius of Earth

$\phi_1 = $ The latitude of the starting point

$\lambda_1 = $ The longitude of the starting point

$\theta = $ The bearing from north

\subsection*{Resources}
\begin{itemize}
    \item \href{https://stackoverflow.com/questions/14263284/create-non-intersecting-polygon-passing-through-all-given-points}{Better Path Generation}
    \item \href{https://www.movable-type.co.uk/scripts/latlong.html}{Haversine Formula}.
\end{itemize}
\end{logbook-entry}

\begin{logbook-entry}{Communication Protocol}{28}{11/03/2024}{15/03/2024}
The communication protocol is worked on.
The main task of the communication protocol is to facilitate communication between the desktop application and the robot boat.
The desktop application would send the route generated by the user to the boat.
The boat would send the data collected to the desktop application for further processing.

\subsection*{Communication Medium}

There are a few ways to send data between the boat and the desktop application.
Some mediums considered were:

\begin{itemize}
    \item Radio Frequency Communication using nRF24L01
    \item Serial Communication through USB Cables
    \item Bluetooth Connection with the Raspberry Pi
    \item Simple TCP Connection using WiFi with the Raspberry Pi
\end{itemize}

\begin{table}[h]
    \centering
    \caption{Table Comparing All the Communication Mediums}
    \label{tab:comm-medium}
    \vspace{1em}
    \begin{tabular}{ p{7em} p{10em} p{10em} }
        Communication Medium & Advantages & Disadvantages\\
        \hline
        RF & Wireless, Does not need connection with the Raspberry Pi  & Slow and Limited Packet Size\\
        \hline
        USB Cable & Simple and Practical & Requires physical access to the boat\\
        \hline
        Bluetooth & Wireless, Easy discovery of robot & Requires Bluetooth connection with the Raspberry Pi, short range\\
        \hline
        TCP & Wireless, Fast data transfers & Requires LAN connection with Pi, medium range, compatible with all laptops\\
        \hline
    \end{tabular}
\end{table}

A comparison between all the communication mediums are shown in table \ref{tab:comm-medium}.
Each of the mediums were carefully considered and compared.
The serial communication using USB cables are chosen at the end.

The RF communication isn't considered due to the low packet sizes and sending speed.
This can be circumvented by splitting the data into multiple packets.

However, sending multiple packets will take time as each packet send has an estimated delay of around 1 ms.

TCP connection isn't considered due to requiring connecting the Raspberry Pi to the same WLAN network (mobile hotspot).
This is hard to achieve without access to the GUI output of Raspberry Pi.

The short range of the Bluetooth protocol limits its usability for our application.
The benefits of being wireless is that data can be transferred while the boat is operating.
The Bluetooth connection would constantly disconnect and or the user needs to stay close to the boat to maintain the connection.

Even though the lack of wireless capabilities of the USB cable, this will not cause too much inconvenience for the user.
Before and after each operation of the boat, the user need to collect/place down the boat.
The user will have physical access to the boat and therefore can use the USB cable.
\end{logbook-entry}

\end{document}
